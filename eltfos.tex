\documentclass{beamer}
\usetheme{Goettingen}
\usecolortheme{wolverine}
%\usefonttheme{serif}

\usepackage{geometry}
\usepackage{amsmath}
\usepackage{amssymb}
\usepackage{bussproofs}
\usepackage[utf8]{inputenc}
\usepackage[english]{babel}
\usepackage{amsthm}
\usepackage{mathtools}
%\usepackage{authblk}
\usepackage{pict2e}
\usepackage[mathscr]{euscript}
\usepackage{tikz-cd}
\tikzstyle{none}=[inner sep=0pt]
\tikzstyle{circ}=[circle,fill=black,draw,inner sep=3pt]
\usepackage{fancybox}
\usepackage[all,2cell]{xy} \UseAllTwocells
\usetikzlibrary{arrows, positioning, intersections}
\input{coya-tikz.tex}
\tikzset{%
	symbol/.style={%
		draw=none,
		every to/.append style={%
			edge node={node [sloped, allow upside down, auto=false]{$#1$}}}
	}
}

% hyperlinks
% \usepackage{color}
% \definecolor{myurlcolor}{rgb}{0.5,0,0}
% \definecolor{mycitecolor}{rgb}{0,0,1}
% \definecolor{myrefcolor}{rgb}{0,0,1}
% \usepackage[pagebackref]{hyperref}
% \hypersetup{colorlinks,
% linkcolor=myrefcolor,
% citecolor=mycitecolor,
% urlcolor=myurlcolor}

% \newcommand{\define}[1]{{\bf \boldmath{#1}}}
% \renewcommand*{\backref}[1]{(Referred to on page #1.)}

% theorems

% \theoremstyle{definition}
% \newtheorem{theorem}{Theorem}
% \newtheorem{definition}[theorem]{Definition}
% \newtheorem{lemma}[theorem]{Lemma}
% \newtheorem{example}[theorem]{Example}
% \newtheorem{corollary}{Corollary}[theorem]

\def\rd{\rotatebox[origin=c]{90}{$\dashv$}} %rotate dash right
\def\ld{\rotatebox[origin=c]{-90}{$\dashv$}} %rotate dash left

% categories

\newcommand{\Th}{\mathsf{Th}}
\newcommand{\Gph}{\mathsf{Gph}}
\newcommand{\Set}{\mathsf{Set}}
\newcommand{\Grp}{\mathsf{Grp}}
\newcommand{\Cat}{\mathsf{Cat}}
\newcommand{\Law}{\mathsf{Law}}
\newcommand{\Mnd}{\mathrm{Mnd}}
\newcommand{\Top}{\mathsf{Top}}
\newcommand{\Mon}{\mathsf{Mon}}
\newcommand{\Alg}{\mathsf{Alg}}
\newcommand{\CCC}{\mathsf{CCC}}
\newcommand{\Pos}{\mathsf{Pos}}
\newcommand{\Mod}{\mathsf{Mod}}
\newcommand{\FinSet}{\mathsf{FinSet}}
\newcommand{\NN}{\mathsf{N}}
\newcommand{\A}{\mathsf{A}}
\newcommand{\V}{\mathsf{V}}
\newcommand{\W}{\mathsf{W}}
\newcommand{\D}{\mathsf{D}}
\newcommand{\C}{\mathsf{C}}
\newcommand{\R}{\mathsf{R}}
\newcommand{\X}{\mathsf{X}}
\newcommand{\K}{\mathsf{K}}
\newcommand{\J}{\mathsf{J}}
\newcommand{\T}{\mathsf{T}}
\newcommand{\Kl}{\mathsf{Kl}}
\newcommand{\LTS}{\mathsf{LTS}}

\newcommand{\FC}{\mathrm{FC}}
\newcommand{\FP}{\mathrm{FP}}
\newcommand{\FS}{\mathrm{FS}}
\newcommand{\UC}{\mathrm{UC}}
\newcommand{\UP}{\mathrm{UP}}
\newcommand{\UG}{\mathrm{UG}}

\newcommand{\op}{\mathrm{op}}
\newcommand{\Obj}{\mathrm{Ob}}

\newcommand{\pic}{$\pi$-calculus}

\newcommand{\pfk}{\pitchfork}
\newcommand{\maps}{\colon}
\newcommand{\id}{\mathrm{id}}

\newcommand{\ul}[1]{\underline{#1}}


\title{Enriched Lawvere Theories\\ for Operational Semantics}
\author{John C. Baez: baez@math.ucr.edu\\ Christian Williams: williams@same}
\institute{University of California, Riverside}
\date{SYCO 4, May 22 2019}

\begin{document}
\frame{\titlepage}

\begin{frame}{Introduction}
  How do we integrate syntax and semantics?
  \[\begin{array}{ll}
      \text{type} & \text{object}\\
      \text{term} & \text{morphism}\\
      \text{rewrite} & \text{2-morphism}\\
    \end{array}\]
  \[\begin{tikzcd}[ampersand replacement=\&]
      (\lambda x.x+x \; \; 2) \ar{r}{\beta} \& 2+2 \ar{r}{+} \& 4
    \end{tikzcd}\]
\end{frame}
\begin{frame}{Change of semantics}
  \[\begin{tikzcd}[column sep=small, ampersand replacement=\&]
\Gph \arrow[bend left,below]{rr}{\FC}
\& \ld \&
\arrow[bend left,above]{ll}{\UG} \Cat \arrow[bend left,below]{rr}{\FP}
\& \ld \&
\arrow[bend left,above]{ll}{\UC} \Pos \arrow[bend left,below]{rr}{\FS}
\& \ld \&
\Set \arrow[bend left,above]{ll}{\UP}
\end{tikzcd}\]

\[\begin{array}{ll}
    \FC_* & \text{maps small-step to big-step operational semantics.}\\
    \FP_* & \text{maps big-step to full-step operational semantics.}\\
    \FS_* & \text{maps full-step to denotational semantics.}\\
\end{array}\]

\end{frame}

\section{theories}
\subsection{Lawvere theories}
\begin{frame}{Lawvere theories}
  A \textbf{Lawvere theory} is a category $\T$ whose objects are finite powers of a distinguished object $t = \tau(1)$.
  $$\tau\maps \mathbb{N}^{\op}\to \T$$
  Morphisms $t^n\to t$ are $n$-ary operations, and commuting diagrams are equations.\\~\\
  Classical algebra is represented and internalized in any category with finite products.
\end{frame}
\begin{frame}{Models}
  Let $\C$ be a category with finite products.\\~\\
  A \textbf{model} of a Lawvere theory $\T$ in $\C$ is a product-preserving functor $\mu\maps \T\to \C$. A morphism of models is a natural transformation. \\~\\
  These form a category $\Mod(\T,\C)$. For example, $\Mod(\Th(\Grp),\Top)$ is the category of topological groups.
\end{frame}
\subsection{theories and monads}
\begin{frame}{Monadicity}
  Another way to describe algebraic structure is by \textit{monads}. Theories induce ``free-forgetful'' adjunctions: 
  \[\begin{tikzcd}[column sep=small, ampersand replacement=\&]
      \Set \arrow[bend left,below]{rr}{F}
      \& {\phantom{AA} \ld} \&
      \arrow[bend left,above]{ll}{U} \Mod(\T,\Set).
    \end{tikzcd}\]
  \[\begin{array}{lll}
      U :: \mu\mapsto \mu(1) && F :: n\mapsto \T(n,-)\\
      && \text{extends to $\Set$ by colimit.}
    \end{array}\]\\
  This induces a monad $T$ on $\Set$, and we have that:
  $$T\text{-Alg} \simeq \Mod(\T,\Set).$$
\end{frame}

\section{enrichment}
\subsection{enriched categories}
\begin{frame}{Enriched categories}
Let $\V$ be a monoidal category. A \textbf{$\V$-enriched category} is:
  \[\begin{array}{ll}
\text{object set} & \Obj(\C)\\
\text{hom function} & \C(-,-)\maps \Obj(\C) \times \Obj(\C) \to \Obj(\V)\\
\text{composition} & \circ_{a,b,c}\maps\C(b,c) \times \C(a,b) \to \C(a,c)\\
\text{identity} & i_a\maps 1_\V \to\C(a,a)\\
    \end{array}\]
  with composition associative and unital.\\
  A \textbf{$\V$-functor} $F\maps \C\to \D$ is:
  \[\begin{array}{ll}
\text{object function} & F\maps \Obj(\C) \to \Obj(\D)\\
\text{hom morphisms} & F_{ab}\maps \C(a,b) \to \D(F(a),F(b))\\
    \end{array}\]
  preserving composition and identity.\\
  A \textbf{$\V$-natural transformation} $\alpha\maps F \Rightarrow G$ is:
\[\begin{array}{ll}
\text{components} & \alpha_a \maps 1_\V \to \D(F(a),G(a))\\
\end{array}\]
 which is ``natural'' in $a$. These form the 2-category \textbf{$\V\Cat$}. 
\end{frame}
\subsection{enriched limits}
\begin{frame}{Enriched limits}
  a
\end{frame}
\begin{frame}{Preservation}
  a
\end{frame}

\section{enriched theories}
\subsection{enriched Lawvere theories}
\begin{frame}{Enriched Lawvere theories}
  a
\end{frame}
\begin{frame}{Generalized arities}
  a
\end{frame}
\subsection{enriched monadicity}
\begin{frame}{Enriched monadicity}
  a
\end{frame}
\subsection{examples}
\begin{frame}{Example: pseudomonoid}
  \[ \Th(\mathrm{PsMon}) \]
  \[\begin{array}{lll}
  \textbf{type}\\ M & \text{pseudomonoid}\\ \\
  \textbf{operations}\\ m\maps M^2 \to M & \text{multiplication}\\
                e\maps1 \to M & \text{identity}\\ \\
  \textbf{rewrites}\\ \alpha \colon m(m \times \id_M) \cong m (\id_M \times m) & \text{associator}\\
                \lambda\maps  m(e \times \id_M) \cong \id_M & \text{left unitor}\\
                \rho\maps m(\id_M \times e) \cong \id_M & \text{right unitor}\\ \\
  \textbf{equations} & \text{pentagon, triangle}
  \end{array}\]
\end{frame}
\begin{frame}{Example: cartesian object}
  
  \[ \Th(\mathsf{Cart}) \]
  \[\begin{array}{lllllll}
      \textbf{type}\\ \X & \text{cartesian object}\\ \\
      \textbf{operations}\\ m \maps \X^2 \to \X & \text{product} \\ 
        e \maps 1 \to \X & \text{terminal element} \\ \\
      \textbf{rewrites}\\
      \bigtriangleup\maps \mathrm{id}_\X \Longrightarrow m \circ\, \Delta_\X & \text{unit of $m \vdash \Delta_\X$}\\
      \pi\maps \Delta_\X \circ m \Longrightarrow \mathrm{id}_{\X^2} & \text{counit of $m \vdash \Delta_\X$}\\
      \top\maps \mathrm{id}_\X \Longrightarrow e \,\circ\, !_\X & \text{unit of $e \vdash !_\X$} \\
      \epsilon \maps !_\X \, \circ \, e \Longrightarrow \mathrm{id}_{1} & \text{counit of $e \vdash !_\X$} \\\\
 \textbf{equations}  & \text{triangle identities}
 \end{array}\]
\end{frame}

\section{natural number arities}
\subsection{arity subcategory}
\begin{frame}{Natural numbers in $\mathsf{V}$}
  a
\end{frame}
\subsection{equivalences}
\begin{frame}{Equivalences}
  a
\end{frame}

\section{change of semantics}
\subsection{change of base}
\begin{frame}{Change of base}
  a
\end{frame}
\subsection{preserving theories}
\begin{frame}{Preservation of theories}
  a
\end{frame}

% \section{grothendieck?}
% \subsection{category of all theories}
% \begin{frame}
  
% \end{frame}

\section{applications}
\subsection{combinators}
\begin{frame}{The $\mathsf{SKI}$-combinator calculus}
  a
\end{frame}
%\subsection{theory}
\begin{frame}{The theory of $\mathsf{SKI}$}
  a
\end{frame}
\subsection{change of base}
\begin{frame}{Change of semantics}
  a
\end{frame}
% \subsection{change of theory}
% \begin{frame}{Change of theory}
%   a
% \end{frame}
% \subsection{bisimulation}
\begin{frame}{Bisimulation}
  a
\end{frame}

\begin{frame}{Conclusion}

\end{frame}

\end{document}